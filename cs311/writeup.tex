\documentclass[letterpaper,10pt,titlepage]{article}

\usepackage{graphicx}                                        

\usepackage{amssymb}                                         
\usepackage{amsmath}                                         
\usepackage{amsthm}                                          

\usepackage{alltt}                                           
\usepackage{float}
\usepackage{color}

\usepackage{url}

\usepackage{balance}
\usepackage[TABBOTCAP, tight]{subfigure}
\usepackage{enumitem}

\usepackage{pstricks, pst-node}

\usepackage{geometry}
\geometry{textheight=9in, textwidth=6.5in}

%random comment

\newcommand{\cred}[1]{{\color{red}#1}}
\newcommand{\cblue}[1]{{\color{blue}#1}}

\usepackage{hyperref}


%pull in the necessary preamble matter for pygments output
\input{pygments.tex}

%% The following metadata will show up in the PDF properties
\hypersetup{
  colorlinks = true,
  urlcolor = black,
  pdfauthor = {Emily Dunham},
  pdfkeywords = {cs311 ``operating systems'' files filesystem I/O},
  pdftitle = {CS 311 Project 3: UNIX File I/O},
  pdfsubject = {Emily Dunham CS 311 Project 3},
  pdfpagemode = UseNone
}

\parindent = 0.0 in
\parskip = 0.2 in

\begin{document}
\tableofcontents

\section{Part 1: System Design}

I planned to work directly from the project instructions, creating a main function to parse user input and then one function per major command. 

\section{Part 2: Work Log}

Due to being overcommitted with various responsibilities this term and afraid of the assignment, I didn't begin  work on the project until Sunday 10/22. On Sunday and Monday, I got the program to the point of being able to parse all necessary user inputs and execute the correct functions. 

I then invoked my grace day because I had a midterm early in the morning on 10/23, and didn't want to fail more than one class this term. After the midterm, I proceeded to attempt to implement the functions for the archiver's various use cases, but my plans were thwarted by an inability to comprehend file I/O. 

After determing that it would be impossible for me to get my program entirely working even with the use of a second grace day, I verified that it at least compiles on os-class in order to hopefully achieve a non-zero score, and then submitted it.


\section{Part 3: Challenges}

The biggest challenge I faced was the fact that I am verifyably stupid: despite having all of the necessary information and  documentation at my disposal, I was unable to create a working archiver program. Additionally, I was humiliated by the prospect of admitting the extent of my stupidity to anyone, so I attemped to do my own work rather than getting help. 

Another  challenge was the fact that Kevin's provided \LaTeX makefile doesn't actually work, and throws such a generic error that troubleshooting the exact cause of the problem is nearly impossible. 

\section{Part 4: Question-Answers}

(a) The main point of the assignment is to encourage us to learn about file I/O, with a secondary purpose about time management for completing large projects. 

(b) I did not ensure my solution's correctness, because I did not complete a working solution for the assignment. 

(c) I learned that I do not have the basic skills required for a successful career in software development. 

%input the pygmentized output of mt19937ar.c, using a (hopefully) unique name
%this file only exists at compile time. Feel free to change that.
\input{__foo.py.tex}
\end{document}

